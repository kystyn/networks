Рассмотрим пример работы программы для линейной топологии с 3 узлами. Здеь и далее: узлы – указаны их номера, связи – список номеров соседних узлов на позиции текущего узла.

\begin{itemize}
	\item Узлы [0, 1, 2]
	\item Связи [[1], [0, 2], [1]]
\end{itemize}

К сети подключены все 3 узла. Кратчайшие пути:
\begin{itemize}
	\item 0: [[0], [0, 1], [0, 1, 2]]
	\item 1: [[1, 0], [1], [1, 2]]
	\item 2: [[2, 1, 0], [2, 1], [2]]
\end{itemize}

От сети отключен 2-ой узел. Новые кратчайшие пути:
\begin{itemize}
	\item 0: [[0], [0, 1], []]
	\item 1: [[1, 0], [1], []]
	\item 2: [[], [], [2]]
\end{itemize}

Теперь рассморим пример работы программы для кольцевой топологии с 3 узлами.
\begin{itemize}
	\item Узлы [0, 1, 2]
	\item Связи [[2, 1], [0, 2], [1, 0]]
\end{itemize}

К сети подключены все 3 узла. Кратчайшие пути:
\begin{itemize}
	\item 0: [[0], [0, 1], [0, 2]]
	\item 1: [[1, 0], [1], [1, 2]]
	\item 2: [[2, 0], [2, 1], [2]]
\end{itemize}

От сети отключен 1-ый узел. Новые кратчайшие пути:
\begin{itemize}
	\item 0: [[0], [], [0, 2]]
	\item 1: [[], [1], []]
	\item 2: [[2, 0], [], [2]]
\end{itemize}


Наконец, рассмотрим пример работы программы для звездной топологии с 4 узлами. Центр в узле с индексом 1.

\begin{itemize}
	\item Узлы [0, 1, 2, 3]
	\item Связи [[1], [0, 2, 3], [1], [1]]
\end{itemize}

От сети отключен 3-ий узел. Новые кратчайшие пути:
\begin{itemize}
	\item 0: [[0], [0, 1], [0, 1, 2], []]
	\item 1: [[1, 0], [1], [1, 2], []]
	\item 2: [[2, 1, 0], [2, 1], [2], []]
	\item 3: [[], [], [], [3]]
\end{itemize}
